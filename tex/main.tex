\documentclass{book}
\usepackage{ucs}
\usepackage[pdftex,plainpages=false,pdfpagelabels,pdfstartpage=1,pdfstartview=FitH]{}
\usepackage[utf8]{inputenc}
\usepackage{polski}
\usepackage[polish]{babel}
\usepackage{setspace}
\usepackage{graphicx}
\usepackage{listingsutf8}
\usepackage{color}

% Original author: Al Sweigart, http://inventwithpython.com/
% License: BY-NC-SA, http://creativecommons.org/licenses/by-nc-sa/3.0/us/
% Translation: GDR!, http://gdr.geekhood.net/

\onehalfspacing

% According to polish grammar rules, you have to put dots after numerals in chapter, section, subsection, etc. headers.
\renewcommand\thechapter{\arabic{chapter}.}
\renewcommand\thesection{\arabic{chapter}.\arabic{section}.}
\renewcommand\thesubsection{\arabic{chapter}.\arabic{section}.\arabic{subsection}.}
\renewcommand\thesubsubsection{\arabic{chapter}.\arabic{section}.\arabic{subsection}.%
                                                           \arabic{subsubsection}.}
% It's customary (depends on publisher) to indent first paragraph in sections and chapters:
\usepackage{indentfirst}

% Fixes a bug in lstlisting
\lstset{extendedchars=\false,inputencoding=utf8x,showspaces=false}
\lstset{literate={ą}{{\k{a}}}1 {ł}{{\l{}}}1 {ń}{{\'n}}1 {ę}{{\k{e}}}1 {ś}{{\'s}}1 {ż}{{\.z}}1 {ó}{{\'o}}1 {ź}{{\'z}}1 {Ą}{{\k{A}}}1 {Ł}{{\L{}}}1 {Ń}{{\'N}}1 {Ę}{{\k{E}}}1 {Ś}{{\'S}}1 {Ż}{{\.Z}}1 {Ó}{{\'O}}1 {Ź}{{\'Z}}1 }

% Numbered listing
\definecolor{codebg}{gray}{0.9}
\newcommand{\bnl}[1]{\lstset{language=Python,numbers=left,backgroundcolor=\color{codebg},frame=single,#1}}

% Non-numbered listing
\newcommand{\bl}[1]{\lstset{language=Python,numbers=none,backgroundcolor=\color{codebg},frame=single,#1}}

% Image macro
\newcommand{\img}[3]{
\begin{figure}
\centerline{
	\includegraphics[width=#3]{#1.png}
}
\caption{#2}
\label{#1}
\end{figure}
}

% Topics covered 
\definecolor{topicscoveredbg}{gray}{0.9}
          
\makeatletter\newenvironment{graybox}{%
\begin{lrbox}{\@tempboxa}\begin{minipage}{\columnwidth}}{\end{minipage}\end{lrbox}%
\colorbox{topicscoveredbg}{\usebox{\@tempboxa}}
}\makeatother

\newcommand{\btopicscovered}{
	\begin{graybox}
	\begin{itemize}
}
\newcommand{\etopicscovered}{
	\end{itemize}
	\end{graybox}
}

%%%%%%%%%%%%%%%%%%%%%%%%%%%%%%%%%%%%%%%%%%%%%%%%%%%%%%%%%%%%%%%%%%%%%%%%%%%%%%%%%%%%%%%%%%%%

\begin{document}

\chapter{Instalowanie Pythona}

\chapter{Shell interaktywny}
\btopicscovered
	\item Liczby całkowite i~zmiennoprzecinkowe
	\item Wyrażenia
	\item Wartości
	\item Operatory
	\item Ewaluacja wyrażeń
	\item Przechowywanie wartości w~zmiennych
	\item Nadpisywanie zmiennych
\etopicscovered

Zanim rozpoczniemy pisać gry komputerowe, powinniśmy poznać kilka pojęć związanych z~programowaniem. Chodzi o~wartości, operatory, wyrażenia i~zmienne. Nie zaczniemy jeszcze w~tym rozdziale pisać programów, ale poznanie tych pojęć i~różnych nazw bardzo uprości dalszą naukę. To dlatego, że większa część programowania to tylko kilka prostych pomysłów połączonych w~całość i~tworzących duże programy.

Zacznijmy od nauczenia się jak używać interaktywnego shella Pythona.

\section{Trochę prostej matematyki}

% Tego akapitu nie było w oryginale
Python posiada interaktywnego shella o~nazwie IDLE.~Interaktywność oznacza że reaguje on na~wszystkie twoje polecenia zaraz po ich wpisaniu (nieinterkatywny tryb pracy to taki w~którym wpierw wpisujesz wszystkie polecenia do~pliku, a~dopiero potem są one wykonywane po kolei). Shell\footnote{Shell jest przez niektórych nazywany również ,,powłoką''. Zabawne, prawda?} to miejsce gdzie wpisujesz polecenia - zgłasza się znakiem \lstinline{>>>} nazywanym znakiem zachęty (lub promptem).

Żeby otworzyć IDLE na~Windowsie, kliknij na~Start, Programy, Python 3.1 a~potem IDLE (Python GUI). Zaczniemy od kilku prostych obliczeń w~Pythonie. Interaktywny shell może działać jak kalkulator. Wpisz w~okno shella 2+2 i~wciśnij Enter na~klawiaturze (na~niektórych klawiaturach ten klawisz to Return). Jak widzisz na~rysunku~\ref{idle-dwaplusdwa}, komputer powinien odpowiedzieć liczbą 4, sumą 2+2.

\img{idle-dwaplusdwa}{Wpisz w~okno shella 2+2}{7 cm}

Jak widzisz, możemy używać shella Pythona jako kalkulatora. To co zrobiłeś nie jest jeszcze programem, ponieważ dopiero uczymy się podstaw. Znak + każe komputerowi dodać liczby 2 i~2. Aby odjąć liczby, użyj znaku -, żeby je pomnożyć użyj gwiazdki (*), i~tak dalej:

\begin{table}[here]
\caption{Różne operatory matematyczne w~Pythonie}
\centerline{
\begin{tabular}{| c | c | }
\hline
2+2 & dodawanie\\
\hline
2-2 & odejmowanie\\
\hline
2*2 & mnożenie\\
\hline
2/2 & dzielenie\\
\hline
\end{tabular}
}
\label{hops-algo}
\end{table}

Znaki +, -, * i~/ kiedy używa się ich w~ten sposób, są nazywane {\bf operatorami} ponieważ mówią komputerowi żeby wykonał jakąś operację na~liczbach które je otaczają.

\subsection{Liczby całkowite i~zmiennoprzecinkowe}

W programowaniu (a~także matematyce), liczby takie jak 4, 0 czy 99 nazywane są {\bf całkowitymi}. Liczby z~ułamkami bądź kropką dziesiętną\footnote{W matematyce do~oddzielania części całkowitej od ułamkowej używa się przecinka, w~programowaniu - zazwyczaj jest to kropka.} (jak 3.5, 42.1, 5.0) nie są całkowite. W~Pythonie, liczba 5 jest całkowita, ale jeśli zapiszemy ją jako 5.0 - już nie. Liczby z~kropką dziesiętną nazywają się {\bf liczbami zmiennoprzecinkowymi}. W~matematyce, 5.0 ciągle pozostaje liczbą całkowitą i~ma tą samą wartość co 5, ale komputer traktuje liczby całkowite i~zmiennoprzecinkowe inaczej.

\subsection{Wyrażenia}

Spróbuj wpisać któreś z~tych działań do~shella, wciskając po każdym z~nich Enter.

\begin{lstlisting}
2+2+2+2+2
8*6
10-5+6
2  +       2      
\end{lstlisting}

\img{idle-wyrazenia}{Okno IDLE po wykonaniu wyrażeń}{7 cm}

Te działania matematyczne nazywane są wyrażeniami. Komputery potrafią rozwiązać miliony tego typu zadań w~ciągu sekundy. Wyrażenia są złożone z~{\bf wartości} (liczb) połączonych {\bf operatorami} (symbolami matematycznymi). Dowiedzmy się czym dokładnie są wartości i~operatory.

\img{idle-wyroperwart}{Wyrażenie składa się z~wartośći i~operatorów}{3 cm}

W ostatnim wyrażeniu w~przykładach wyżej mogłeś się przekonać, że można użyć dowolnej ilości spacji (odstępów) pomiędzy liczbami oraz operatorami. Pamiętaj jednak, żeby zawsze zaczynać pisać od samego początku linii, nie wstawiając dodatkowych odstępów przed pierwszą liczbą.

Liczba jest rodzajem wartości. Liczba całkowita jest rodzajem liczby. Ale, chociaż liczby całkowite są liczbami, nie wszystkie liczby są liczbami całkowitymi (na~przykład, ułamki i~liczby z~kropką dziesiętną jak 2.5 są liczbami ale nie są całkowite).

To podobnie jak kot jest rodzajem zwierzęcia domowego, ale nie wszystkie zwierzęta domowe są kotami. Są przecież ludzie którzy mają psy lub kraby. 

{\bf Wyrażenie} składa się z~wartości (na~przykład liczb całkowitych takich jak 8 czy 6) połączonych operatorem (takim jak znak mnożenia *). Pojedyncza wartość bez operatorów też traktowana jest jak wyrażenie.

W następnym rozdziale nauczymy się jak pracować z~wyrażeniami tekstowymi. Python nie jest ograniczony do~liczb. Umie wiele więcej niż zwykły kalkulator.

\section{Ewaluacja wyrażeń}

Kiedy komputer rozwiązuje wyrażenie 10+5 i~otrzymuje wartość 15, mówimy że {\bf ewaluuje} (oblicza) wyrażenie. Ewaluacja wyrażenia zamienia je w~pojedynczą wartość, podobnie jak rozwiązanie zadania matematycznego zamienia je w~pojedynczą liczbę: rozwiązanie.

Wyrażenia 10+5 oraz 10+3+2 mają tą samą wartość, ponieważ oba są ewaluowane do~15. Nawet pojedyncze wartości są wyrażeniami: wyrażenie 15 ewaluuje się do~wartośći 15.

Jeśli jednak wpiszesz w~interaktywnym shellu tylko 5+, dostaniesz wiadomość o~błędzie.
\lstset{language=python}
\begin{lstlisting}{}
>>> 5 +
SyntaxError: invalid syntax
\end{lstlisting}

Ten błąd zdarzył się ponieważ 5+ nie jest wyrażeniem. Wyrażenia to wartości połączone operatorami, a~operator + zawsze spodziewa się dwóch wartości. My podaliśmy mu tylko jedną. Dlatego właśnie pojawił się komunikat o~błędzie. Błąd składni (Syntax Error) oznacza, że komputer nie rozumie instrukcji którą mu podałeś ponieważ wpisałeś ją nieprawidłowo. Python zawsze wyświetli błąd jeśli wprowadzisz instrukcję której nie zrozumie.

To może nie wydawać się w~tej chwili ważne, ale duża część programowania polega nie tylko na~mówieniu komputerowi co ma zrobić, ale też na~posiadaniu wiedzy jak dokładnie powiedzieć komputerowi żeby to zrobił.

\subsection{Wyrażenia wewnątrz innych wyrażeń}

% Różni się od oryginału - krótsze zdania.
Wyrażenia mogą również zawierać inne wyrażenia. Na przykład, wyrażenie 2+5+8 zawiera dwa operatory +, co powoduje że składa się z~dwóch mniejszych wyrażeń.~Python najpierw oblicza 2+5, wynikiem jest 7. Następnie oblicza 7+8 i~wynikiem jest 15.

Pomyśl o~wyrażeniu jak o~stosie cukierków: jeśli położysz dwa stosy cukierków jeden na~drugim, będziesz miał znów stos cukierków (tyle, że większy). Duży stos cukierków może być złożony z~mniejszych stosów wysypanych jeden na~drugi. Wyrażenia mogą być składane razem i~w~ten sam sposób tworzyć większe wyrażenia. Ale niezależnie od tego jak duże jest wyrażenie, zawsze będzie się ewaluowało do~pojedynczej odpowiedzi, podobnie jak 2+5+8 ewaluuje się do~15.

\section{Przechowywanie wartości w~zmiennych}

Kiedy piszemy program, często chcielibyśmy zachować wyniki ewaluacji naszych wyrażeń żeby móc ich używać w~dalszej części programu. Możemy przechowywać wartości w~{\bf zmiennych}.

Pomyśl że zmienna to pudełko które może przechowywać wartość. Możesz przechować wartość w~zmiennej używając znaku = (nazywanego {\bf operatorem przypisania}). na~przykład, żeby przechować wartość 15 w~zmiennej o~nazwie ,,liczba'', wpisz w~shellu liczba=15:

\lstset{language=python}
\begin{lstlisting}{}
>>> liczba = 15
>>>
\end{lstlisting}

\img{idle-zmienna}{Zmienne są jak pudełka które przechowują wartości}{4 cm}

Zmienna którą stworzyłeś jest jak pudełko z~wartością 15 w~środku (jak na~rysunku \ref{idle-zmienna}). Nazwa zmiennej (,,liczba'') jest jak etykieta na~pudełku (żeby można było rozróżnić jedną zmienną od drugiej), a~wartość przechowywana przez zmienną jest jak karteczka w~środku pudełka.

Kiewy wciśniesz Enter, nie zobaczysz żadnej odpowiedzi od Pythona. O~ile nie zobaczysz wiadomości o~błędzie, możesz założyć że instrukcja została wykonana poprawnie. Pojawi się następny znak zachęty \lstinline{>>>} żebyś mógł wpisać kolejną instrukcję.

Ta instrukcja (nazywana {\bf instrukcją przypisania}) tworzy zmienną ,,liczba'' i~przechowuje w~niej wartość 15. W~przeciwieństwie do~wyrażeń, {\bf instrukcje} nie są ewaluowane do~żadnej wartości. Dlatego właśnie shell nie wyświetlił żadnej wartości w~odpowiedzi.

Może nie być łatwo zapamiętać które polecenia są wyrażeniami a~które instrukcjami. Po prostu pamiętaj że jeśli polecenie ewaluuje się do~jakiejś wartości, jest wyrażeniem. Jeśli nie, jest instrukcją.

Instrukcja przypisania jest zapisana w~następujący sposób: nazwa zmiennej, później znak = oraz wyrażenie. Wartość do~której ewaluuje się to wyrażenie jest przechowywana w~zmiennej. Wartość 15 sama w~sobie jest wyrażeniem. Wyrażenia złożone z~pojedynczej wartości są proste do~ewaluacji. Takie wyrażenia ewaluują się po prostu do~tej samej wartości. Na przykład, wyrażenie 15 ewaluuje się do~wartości 15.

Pemiętaj, zmienne przechowują wartości, nie wyrażenia. Na przykład, jeśli mielibyśmy wyrażenie liczba=10+5, to wyrażenie 10+5 zostałoby wpierw obliczone, a~następnie wynik (15) zostałby przechowany w~zmiennej ,,liczba''.

Kiedy zapisujesz wartość do~zmiennej po raz pierwszy używając instrukcji przypisania, Python stworzy tą zmienną. Za każdym następnym razem, operator przypisania zamieni wartość zapamiętaną w~tej zmiennej.

Sprawdźmy czy zmienna stworzyła się poprawnie. Jeśli wpiszemy w~shellu tylko ,,liczba'', powinniśmy zobaczyć jaka wartość jest przechowywana w~zmiennej o~tej nazwie:

\begin{lstlisting}
>>> liczba = 15
>>> liczba
15
>>> 
\end{lstlisting}

W tej chwili wyrażenie ,,liczba'' jest ewaluowane do~wartości przechowywaną przez zmienną ,,liczba'', czyli 15.

Wiedząc to wszystko, możemy zrobić ciekawą rzecz. Jeśli teraz wpiszemy do~shella ,,liczba + 5", otrzymamy wartość 20:

\begin{lstlisting}
>>> liczba+5
20
>>> 
\end{lstlisting}

Może się to wydawać dziwne, ale ma sens jeśli przypomnimy sobie że ustawiliśmy wartość ,,liczba'' na~15. Z~tego powodu, napisanie ,,liczba + 5'' jest zupełnie jak napisanie ,,15 + 5".

Jeśli spróbujesz użyć zmiennej zanim została stworzona, Python zwróci ci komunikat błędu ponieważ taka zmienna jeszcze nie istnieje. Zdarzy się to również jeśli zrobisz literówkę w~nazwie istniejącej zmiennej:


\begin{lstlisting}
>>> lcibza

Traceback (most recent call last):
  File "<pyshell#10>", line 1, in <module>
      lcibza
NameError: name 'lcibza' is not defined
>>> 
\end{lstlisting}

Możemy zmienić wartość przechowywaną w~zmiennej wprowadzając jeszcze jedną instrukcję przypisania. Spróbuj na~przykład tego:

\begin{lstlisting}
>>> liczba=15
>>> liczba + 5
20
>>> liczba = 3
>>> liczba + 5
8
>>> 
\end{lstlisting}

Za pierwszym razem kiedy wpisałeś ,,liczba + 5", wartością wyrażenia było 20, ponieważ zmienna ,,liczba'' przechowywała wartość 15. Ale kiedy wpisałeś ,,liczba = 3", wartość 15 została zamieniona (nadpisana) wartością 3. Z~tego powodu, jeśli wpisałeś później ,,liczba + 5", wartością wyrażenia było 8.

Aby dowiedzieć się jaką wartość przechowuje zmienna, wpisz jej nazwę w~shellu.

Ponieważ zmienna jest tylko nazwą dla wartości, możemy robić też takie działania jak poniżej:

\begin{lstlisting}
>>> liczba = 15
>>> liczba + liczba
30
>>> liczba - liczba
0
>>> 
\end{lstlisting}

Kiedy zmienna ,,liczba'' przechowuje wartość 15, wpisanie ,,liczba + liczba'' odpowiada wpisaniu ,,15 + 15". Wartością takiego wyrażenia jest 30. Z~kolei wyrażenie ,,liczba - liczba'' ma wartość 0 ponieważ odpowiada wyrażeniu ,,15 - 15". Wyrażenia powyżej używają zmiennej ,,liczba'' dwa razy. Możesz używać zmiennych w~wyrażeniach ile tylko razy chcesz. Pamiętaj że Python będzie ewaluował nazwę zmiennej do~wartości w~niej przechowywanej za każdym razem kiedy zmienna będzie użyta.

Możemy nawet użyć wartości zmiennej ,,liczba'' aby przypisać zmiennej ,,liczba'' nową wartość!

\begin{lstlisting}
>>> liczba = 15
>>> liczba = liczba + 5
>>> liczba
20
\end{lstlisting}

Wyrażenie przypisania ,,liczba = liczba + 5'' jest jak powiedzenie ,,nowa wartość zmiennej liczba będzie równa obecnej wartości plus pięć''. Pamiętaj że do~zmiennej po lewej stronie znaku = będzie przypisana wartość do~której ewaluuje się wyrażenie po prawej stronie znaku. Możemy też zwiększać wartość zmiennej liczba kilka razy:

\begin{lstlisting}
>>> liczba = 15
>>> liczba = liczba + 5
>>> liczba = liczba + 5
>>> liczba = liczba + 5
>>> liczba
30
\end{lstlisting}

\section{Używanie więcej niż jednej zmiennej}

Kiedy piszesz program, nie jesteś ograniczony do~używania jednej zmiennej. Często będziesz potrzebował używać wielu.

Na przykład, przypiszmy różne wartości do~dwóch zmiennych o~nazwach ,,foo'' oraz ,,bar'':

\begin{lstlisting}
>>> foo = 10
>>> bar = 15
\end{lstlisting}

Teraz zmienna foo ma wartość 10, a~zmienna bar - 15.

\img{idle-zmienne}{Zmienne ,,foo'' i~,,bar'' przechowują różne wartości}{10 cm}

Spróbujmy zmienić teraz zmienną ,,liczba''. Wpisz ,,liczba = foo + bar'' do~okna shella i~sprawdź jaka będzie nowa wartość zmiennej ,,liczba''. Czy potrafisz zgadnąć jaka?

\begin{lstlisting}
>>> foo = 10
>>> bar = 15
>>> liczba = foo + bar
>>> liczba
25
\end{lstlisting}

Wartość zmiennej ,,liczba'' wynosi teraz 25, ponieważ dodając ,,foo'' i~,,bar'', dodajemy wartości przechowywane w~,,foo'' (10) i~,,bar'' (15).

\subsection{Nadpisywanie zmiennych}

Zmiana wartości przechowywanej w~zmiennej jest prosta. Po prostu przypisz jej nową wartość. Sprawdź, co się stanie po wprowadzeniu następującego kodu do~shella:

\begin{lstlisting}
>>> foo = 42
>>> foo
42
>>> foo = 'Czesc'
>>> foo
'Czesc'
\end{lstlisting}

Na początku, zmienna ,,foo'' przechowywała liczbę 42. Dlatego właśnie za pierwszym razem po wpisaniu ,,foo'' komputer zwrócił ,,42". Ale po wykonaniu ,,spam='Czesc'", wartość 42 została usunięta ze zmiennej i~zapomniana w~momencie w którym tekst ,,Czesc'' został w~niej zapamiętany.

Zmiana wartości zmiennej na~nową wartość nazywana jest {\bf nadpisywaniem}. Ważne jest żeby wiedzieć, że stara wartość zostanie na~zawsze zapomniana. Jeśli chcesz ją zapamiętać żeby użyć jej później, możesz zapamiętać ją w~innej zmiennej przed zmianą wartości.

\begin{lstlisting}
>>> foo = 42
>>> foo
42
>>> stareFoo = foo
>>> foo = 'Czesc'
>>> foo
'Czesc'
>>> stareFoo
42
\end{lstlisting}

W powyższym przykładzie, przed nadpisaniem wartości zmiennej ,,foo'', skopiowaliśmy jej wartość do~zmiennej o~nazwie ,,stareFoo''. W~tym momencie, zarówno ,,foo'' jak i~,,stareFoo'' mają wartość 42. W~następnej linii, w~zmiennej ,,foo'' zapamiętany zostaje tekst ,,Czesc'', ale zmienna stareFoo pozostaje nietknięta.

\section{Podsumowanie}

W tym rozdziale nauczyłeś się podstaw pisania instrukcji Pythona. Musisz mówić Pythonowi bardzo dokładnie i~ściśle co ma zrobić ponieważ komputery nie potrafią myśleć i~rozumieją tylko bardzo proste rozkazy. Nauczyłeś się, że Python potrafi ewaluować wyrażenia (to znaczy, zamienić całe wyrażenie na~pojedynczą wartość), i~że wyrażenia to wartości (takie jak 2 czy 5) połączone operatorami (takimi jak + lub -). Dowiedziałeś się także, że możesz przechowywać wartości w~zmiennych, żeby twój program je zapamiętał a~ty żebyś mógł użyć ich później.

W następnym rozdziale poznasz jeszcze kilka podstawowych pojęć, a~później napiszesz swój pierwszy program!

\chapter{Tekst}

\btopicscovered
	\item Wykonanie programu
	\item Stringi
	\item Łączenie stringów
	\item Typy danych (takie jak stringi czy liczby całkowite)
	\item Używanie IDLE do~pisania kodu źródłowego
	\item Zapisywanie i~uruchamianie programów w~IDLE
	\item Funkcja print()
	\item Funkcja input()
	\item Komentarze
	\item Wielkie litery w~nazwach zmiennych
	\item Rozróżnianie wielkości liter
\etopicscovered

Wystarczy już liczb i~matematyki. Python jest znacznie więcej niż tylko kalkulatorem. Sprawdźmy teraz co potrafi robić z~tekstem. W~tym rozdziale nauczymy się jak przechowywać tekst w~zmiennych, łączyć teksty i~wyświetlać je na~ekranie. Wiele z~naszych programów będzie używało tekstu do~wyświetlania gier na~ekranie gracza, a~gracz będzie wprowadzał do nich tekst za pomocą klawiatury. Napiszemy też pierwszy program, który pozdrowi użytkownika tekstem ,,Witaj, świecie!'' i~spyta go o~imię.

\section{Stringi}

W Pythonie używamy małych fragmentów tekstu nazywanych {\bf stringami}\footnote{Niektórzy nazywają je także łańcuchami tekstowymi}. Możemy przechowywać stringi w~zmiennych dokładnie tak, jak przechowywaliśmy liczby. Kiedy wpisujemy stringi, otaczamy je dwoma apostrofami (') w~ten sposób:

\begin{lstlisting}
>>> foo = 'czesc'
>>>
\end{lstlisting}

Apostrofów używa się tylko po to, żeby komputer wiedział gdzie tekst się zaczyna i~kończy. Nie są one częścią tekstu.

% This was not in the original
{\bf Uwaga} - na klawiaturze są dwa znaki apostrofa, jeden w lewym górnym rogu (backtick), drugi - bliżej Entera, na prawo od klawisza ; (właściwy apostrof). My używamy tego drugiego (bliżej Entera).

Jeśli teraz wpiszesz w~shellu \lstinline{foo}, powinienneś zobaczyć zawartość zmiennej \lstinline{foo} (string 'czesc'). 


\begin{lstlisting}
>>> foo = 'czesc'
>>> foo
'czesc'
>>>
\end{lstlisting}

Stało się tak dlatego, że Python ewaluuje zmienną \lstinline{foo} do~wartości przez nią przechowywanej (w~tym wypadku, do~tekstu 'czesc').

Stringi mogą zawierać prawie każdy znak na~klawiaturze (nie mogą zawierać apostrofów bez wcześniejszego ich wyescapowania. Escapowanie będzie opisane dalej). Poniżej są przykłady prawidłowych stringów:

\begin{lstlisting}
'czesc'
'Jak się masz?'
'ALAMAKOTA'
'7 jabłek, 14 pomarańczy, 3 cytryny i czosnek'
'Słuchaj, jeszcze chwila i otworzę wytwórnię salami.'
'Dawno, dawno temu w odległej galaktyce...'
'O*&#wY%*&OCfsdYO*&gfC%YO*&%3yc8r2'
\end{lstlisting}

Podobnie jak w~poprzednim rozdziale robiliśmy to z~liczbami, stringi również możemy łączyć operatorami tworząc wyrażenia.

\section{Konkatenacja stringów}

Możesz ,,dokleić" jednego stringa na~koniec drugiego używając operatora +. Operacja taka nazywa się konkatenacją (łączeniem) stringów. Spróbuj wpisać w~shellu \lstinline{'Witaj' + 'Świecie!'}

\begin{lstlisting}
>>> 'Witaj' + 'Swiecie!'
'WitajSwiecie!'
>>>
\end{lstlisting}

Żeby oddzielić słowa, dopisz spację po stringu 'Witaj' przed zamykającym apostrofem, w~ten sposób:

\begin{lstlisting}
>>> 'Witaj ' + 'Swiecie!'
'Witaj Swiecie!'
>>> 
\end{lstlisting}

Operator + działa w~inny sposób dla stringów i~dla liczb, ponieważ są one różnymi {\bf typami danych}. Każda wartość posiada typ (jest jakiegoś typu). Wartość 'Cześć' jest typu string. Wartość 5 jest typu liczba całkowita. Typ wartości mówi nam (i~komputerowi) jaki rodzaj danych ona przechowuje. 

\section{Pisanie programów w~edytorze IDLE}

Do teraz pisaliśmy instrukcje jedna po drugiej w~interaktywnym shellu. Ale kiedy piszemy program, wpierw wpisujemy kilka instrukcji, a~następnie uruchamiamy je wszystkie razem. Napiszmy więc pierwszy program!

Program który pokazywał okno interaktywnego shella w~którym działaliśmy do~tej pory nazywał się IDLE (Interactive DeveLopment Environment - Interaktywne Środowisko Programistyczne). IDLE ma także inną funkcję - edytor plików.

Kliknij na~menu File (Plik) na~górze okna, wybierz New Window (Nowe Okno). Pojawi sie nowe puste okno w~którym będziesz mógł pisać program. To okno to {\bf edytor plików}.

\img{strings-fileeditor}{Okno edytora plików}{7 cm}

\section{Witaj Świecie!}

Istnieje w programowaniu tradycja polegająca na tym, że kiedy uczysz się nowego języka programowania, piszesz program wyświetlający na ekranie tekst ,,Witaj, Świecie!'' (,,Hello, World!"). Stwórzmy więc własny program Hello World w Pythonie!

% The following 2 paragraphs in original order make little sense
Wprowadź tekst poniżej do~nowego okna edytora plików. Nazywamy ten tekst kodem źródłowym programu ponieważ zawiera instrukcje które Python będzie wykonywał żeby wiedzieć dokładnie jak program powinien się zachować.

Kiedy wprowadzasz program, nie wpisuj liczb po lewej stronie kodu. Są tam jedynie po to, żeby omawiając kod można było się odnieść do~konkretnej linii po numerze. Jeśli spojrzysz na~prawy dolny róg edytora, zobaczysz numer linii w~której kursor znajduje się w~danym momencie.

\img{strings-nrlinii}{Prawy dolny róg okna edytora powie ci w~której linii znajduje się kursor. Tutaj znajduje się w~linii 12.}{3 cm}


{\bf Uwaga!} Ten program działa tylko z~Pythonem w~wersji 3, więc jeśli masz zainstalowanego Pythona 2.6 lub jakąkowiek inną wersję z~serii 2.x, program nie będzie działał. Świadczy o~tym błąd: \lstinline{NameError: name is not defined}.

\bnl{caption=hello.py}
\lstinputlisting{python/hello.py}
% TODO URL

Okno programu IDLE będzie pokazywało różne rodzaje instrukcji w~różnych kolorach. Kiedy skończysz przepisywać kod, okno powinno wyglądać tak, jak na~rysunku \ref{strings-idle}.

\img{strings-idle}{Okno edytora plików powinno po wprowadzeniu kodu  wyglądać w~ten sposób}{10 cm}

\subsection{Zapisywanie programu}

\img{strings-saveas}{Zapisywanie programu}{7 cm}

Po przepisaniu kodu źródłowego programu, zapisz go żebyś nie musiał go wpisywać od nowa za każdym razem kiedy startujesz IDLE.~Aby to zrobić, w~menu File (Plik) wybierz opcję Save As (Zapisz Jako). Powinno pokazać się okno wyboru pliku. Wpisz jako nazwę pliku ,,hello.py'' (jak na~rysunku \ref{strings-saveas}) i~wciśnij Save (Zapisz).

Kiedy wpisujesz program, powinenneś raz na~jakiś czas zapisywać jego kod. W~ten sposób, jeśli komputer przypadkowo się wyłączy lub niechcąco wyjdziesz z~IDLE, zgubi się tylko część tekstu. Wciskając Ctrl-S, szybko zapiszesz plik bez użycia myszki.

%TODO mention about python2?

\subsection{Otwieranie zapisanych programów}

Aby wczytać zapisany program, wybierz w~menu File (Plik) opcję Open (Otwórz). W~oknie które się pojawi, wybierz ,,hello.py'' i~wciśnij przycisk Open (Otwórz). Program hello.py który zapisałeś otworzy się w~oknie edytora plików.

A teraz - czas na~uruchomienie programu! Z~menu Run (Uruchom) wybierz opcję Run Module (Uruchom Moduł) lub po prostu wciśnij na~klawiaturze klawisz F5. Twój program powinien uruchomić się w~oknie shella (tym, które pokazało się kiedy uruchomiłeś IDLE). Pamiętaj, musisz wcisnąć F5 w~oknie edytora plików (tam, gdzie przepisałeś kod) - nie w~oknie interaktywnego shella.

\img{strings-hello}{Tak powinno wyglądać okno shella po uruchomieniu programu ,,Hello World''}{7 cm}

Kiedy program spyta cię o~imię, wpisz je a~potem wciśnij Enter jak na~rysunku \ref{strings-hello}.

\section{Jak działa program Witaj Świecie}

W jaki sposób nasz program działa? Każda z~linii którą wprowadziliśmy jest instrukcją którą Python tłumaczy tak, żeby komputer potrafił ją wykonać. Program komputerowy jest jak przepis kucharski. Najpierw wykonaj pierwszy krok, potem drugi i~tak dalej, aż dotrzesz do~końca. Każda instrukcja jest wykonywana jedna po drugiej, zaczynając od góry programu i~idąc dalej instrukcja po instrukcji. Po tym jak program wykona instrukcje z~pierwszej linii, zaczyna wykonywać drugą linię, potem trzecią i~tak dalej.

Przechodzenie programu od instrukcji do~instrukcji nazywamy {\bf przepływem wykonania} lub po prostu {\bf wykonywaniem}.

Spojrzmy teraz na~program linia po linii żeby zobaczyć co robi. Zacznijmy od pierwszej.

\subsection{Komentarze}
\bnl{caption=,name=hello}
\lstinputlisting[firstline=1,lastline=1]{python/hello.py}

Ta linia nazywana jest {\bf komentarzem}. Każdy tekst za znakiem \lstinline{#} (znak ten nazywany jest {\bf hashem}\footnote{czyt. hasz}) jest komentarzem. Komentarze nie są dla komputera, ale ciebie - programisty. Komputer je ignoruje. Używane są do~przypominania ci co program robi, lub do~mówienia innym ludziom którzy mogą oglądać twój kod co próbujesz w~nim osiągnąć.

Programiści zazwyczaj dodają komentarz na~samym początku kodu żeby nadać programowi tytuł. Program IDLE wyświetla komentarze na~czerwono żeby bardziej się wyróżniały.

\subsection{Funkcje}

{\bf Funkcja} jest rodzajem mini-programu wewnątrz twojego programu. Zawiera linie kodu który jest wykonywany od góry do~dołu. Python dostarcza kilku wbudowanych funkcji, których możemy używać. Najlepszą rzeczą w~funkcjach jest to, że wystarczy że wiemy co dana funkcja robi, nie jak to robi. Jedyne co musisz wiedzieć o~funkcji print() to to, że wyświetla tekst na~ekranie, ale nie musisz wiedzieć jak to robi.

{\bf Wywołanie funkcji} jest fragmentem kodu który mówi naszemu programowi żeby uruchomił kod wewnątrz funkcji. Na przykład, twój program może wywołać funkcję print() kiedykolwiek chcesz wyświetlić tekst na~ekranie. Funkcja print() pobiera stringa który wpisujesz jej pomiędzy nawiasami i~drukuje tekst na~ekranie. ponieważ chcemy wyświetlić na~ekranie tekst Witaj Świecie, piszemy nazwę funkcji, otwierający nawias, string 'Witaj Świecie!' i~nawias zamykający.

\subsection{Funkcja print()}

\bnl{name=hello}
\lstinputlisting[firstline=2,lastline=3,firstnumber=2]{python/hello.py}

Te linie wywołują funkcję {\bf print}, zazwyczaj zapisywaną jako print() (string który chcemy wydrukować na~ekranie jest umieszczony pomiędzy nawiasami).

Nawiasy do~końca nazwy funkcji dodaje się po to, żeby było wiadomo że mówimy o~funkcji o~nazwie print(), nie zmiennej print. Nawiasy na~końcu funkcji przypominają nam że mówimy o~funkcji tak samo, jak cudzysłowie dookoła '42' przypominają nam że mówimy o~stringu '42', nie o~liczbie 42.

Linia 3 to kolejne wywołanie funkcji print(). Tym razem, program pisze ,,Jak się nazywasz?".

\subsection{Funkcja input()}

\bnl{name=hello}
\lstinputlisting[firstline=4,lastline=4,firstnumber=4]{python/hello.py}

Ta linia zawiera instrukcję przypisania, zmienną (imie) i~wywołanie funkcji (input()). Kiedy wywoływane jest input(), program czeka na~tekst od użytkownika. Tekst który wprowadzi użytkownik (czyli twoje imię) staje się wartością zwracaną przez funkcję.

Podobnie jak wyrażenia, wywołania funkcji są ewaluowane do~pojedynczej wartości. Wartość do~której wywołanie funkcji jest ewaluowane nazywa się {\bf wartością zwracaną}. (W~zasadzie, możemy używać słowa ,,zwraca'' zamiast ,,ewaluuje''.) W~tym wypadku, wartość zwracana przez funkcję input() to string który zawiera imię wpisane przez użytkownika. Jeśli użytkownik wpisał Albert, wywołanie funkcji input() jest ewaluowane do~stringa 'Albert'.

Funkcja o~nazwie input() nie potrzebuje (w~przeciwiństwie do~print()) od programu żadnych danych, dlatego nie umieściliśmy nic pomiędzy jej nawiasami.

\bnl{name=hello}
\lstinputlisting[firstline=5,lastline=5,firstnumber=5]{python/hello.py}

W ostatniej linii mamy znów funkcję print(). Tym razem, używamy operatora plus (+) aby skonkatenować stringa 'Miło mi cię poznać, ' i~stringa przechowywanego w~zmiennej o~nazwie \lstinline{imie}, która przechowuje imię które użytkownik wprowadził do~programu. W~ten sposób nauczyliśmy program pozdrawiać nas po imieniu.

\subsection{Kończenie programu}
Kiedy program wykona ostatnią linię, zatrzymuje się. W~tym momencie jest {\bf zakończony} bądź {\bf wyszedł} i~wszelkie zmienne są zapominane przez komputer, włączając stringa przechowywanego w~\lstinline{imie}. Jeśli spróbujesz uruchomić program ponownie podając inne imię, na~przykład Gosia, pomyśli że to twoje imię.

\bl{}
\begin{lstlisting}
Witaj swiecie!
Jak sie nazywasz??
Gosia
Milo mi cie poznac, Gosia
>>> 
\end{lstlisting}

Pamiętaj, komputer robi tylko dokładnie to, co mu powiesz. Tutaj, w~naszym pierwszym programie, jest zaprogramowany by spytać cię o~imię, pozwala ci wpisać tekst, a~potem mówi cześć i~wyświetla tekst który wpisałeś.

Ale komputery są głupie. Program nie ma pojęcia czy wpisujesz swoje imię, imię kogoś innego, czy po prostu coś głupiego. Możesz wpisać cokolwiek chcesz i~komputer potraktuje to w~ten sam sposób:

\bl{}
\begin{lstlisting}
Witaj swiecie!
Jak sie nazywasz??
kupa
Milo mi cie poznac, kupa
>>> 
\end{lstlisting}

\section{Nazwy zmiennych}

Komputerowi jest wszystko jedno jak nazwiesz zmienne, ale tobie nie powinno to być obojętne. Dawanie zmiennym nazw mówiących o~tym, co przechowują sprawia że łatwiej zrozumieć co program robi. Zamiast \lstinline{imie}, mogliśmy nazwać zmienną jaroslawKaczynski lub iMiE.~Komputer uruchomi program tak samo (o~ile nadal będziesz używał nazwy jarosławKaczyński lub iMiE).

Nazwy zmiennych, podobnie jak wszystko inne w~Pythonie, rozróżniają pomiędzy małymi i~wielkimi literami (ang. Case-sensitive). Oznacza to, że ta sama nazwa zmiennej pisana inną wielkością liter uważana jest za zupełnie inną zmienną. Zatem foo, FOO, Foo i~fOO są w~Pythonie czterema różnymi zmiennymi. Każda z~nich może przechowywać inną wartość.

Nie jest najlepszym pomysłem trzymanie zmiennych o~tej samej nazwie różniących się tylko wielkością liter. Jeśli przechowujesz imię użytkownika w~zmiennej \lstinline{IMIE}, a~imię jego psa w~zmiennej \lstinline{imie}, parę tygodni później czytając swój kod nie będziesz wiedział o~co chodzi. Czy \lstinline{IMIE} oznacza imię psa, czy człowieka?

Jeśli przez przypadek zamienisz miejscami zmienne \lstinline{imie} i~\lstinline{IMIE}, twój program będzie działał (to znaczy, nie będzie miał błędów składni), ale będzie się zachowywał nieprawidłowo (zwracając się do~użytkownika per Fafik). Ten typ błędu w~kodzie to {\bf bug}. Często zdarza się przez przypadek pozostawiać w~programie bugi podczas ich pisania. Dlatego właśnie ważne jest, żeby wybierać sensowne nazwy dla zmiennych.

Jeśli nazwa zmiennej składa się z~więcej niż jednego słowa, dobrze jest zaczynać kolejne słowa od wielkiej litery. Jeśli przechowujesz string mówiący o~tym co zjadłeś na~śniadanie, nazwa zmiennej \lstinline{coJadlemNaSniadanie} będzie znacznie łatwiejsza do~odczytania niż \lstinline{cojadlemnasniadanie}. W~programowaniu to {\bf konwencja} - reguła której nie musisz przestrzegać, ale dla własnego dobra powinienneś. (Jeszcze lepiej byłoby nazwać zmienną prościej - dzisiejszeSniadanie). Używanie wielkiej litery na~początku każdego słowa sprawia, że program jest czytelniejszy.

% Paragraph about not using national characters in names - not in original
Zauważyłeś pewnie, że popełniłem błąd w~nazwie zmiennej - słowo imię pisze się z~,,ę'' na~końcu, nie ,,e''. Zrobiłem to jednak celowo. Chociaż możesz używać nazw zmiennych i~funkcji z~polskimi literami (takimi jak ą, ć, ę, ó, ł, ż, ź, ń, ś), najlepiej zastępuj je ich angielskimi odpowiednikami (a, c, e, o, l itp.). Python w~najnowszej wersji radzi sobie z~takimi nazwami dość dobrze, ale nie wszyscy używają najnowszych wersji. Kiedyś prawdopodobnie nauczysz się innego niż Python języka programowania - większość z~nich nie pozwala na~polskie znaki w~nazwach zmiennych i~funkcji, będzie ci ciężko zmienić przyzwyczajenia.

\section{Podsumowanie}

Teraz wiesz już jak radzić sobie z~tekstem, więc możemy zacząć tworzyć programy które użytkownik może uruchamiać i~komunikować się z~nimi. To ważne, ponieważ tekst jest głównym sposobem w~który użytkownik będzie się komunikował z~komputerem. Gracz będzie wprowadzał tekst na~ekran poprzez klawiaturę z~pomocą funkcji input(), a~komputer będzie wyświetlał tekst kiedy będzie wykonywana funkcja print().

Stringi są po prostu jeszcze jednym typem danych, który możemy używać w~programach. Możemy używać operatora + do~złączenia stringów razem. Używanie operatora + do~konkatenacji dwóch stringów i~stworzenia nowego stringa działa tak samo jak użycie operatora + do~dodania dwóch liczb, z~których uzyskujemy nową liczbę (sumę).

W następnym rozdziale, nauczymy się więcej o~zmiennych i~nasz program będzie zapamiętywał tekst i~liczby wprowadzone przez gracza. Wiedząc jak używać tekstu, liczb i~zmiennych, będziemy gotowi żeby zacząć tworzyć gry.

\end{document}
